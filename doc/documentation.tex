% Documentation for laser-propagation
\documentclass{article}
\usepackage[margin=2.5cm]{geometry} % set smaller margins
\usepackage{graphicx}
\usepackage{listings}
\usepackage{xcolor}
\usepackage{courier}
\usepackage{amsmath}
\usepackage{hyperref}

\lstset{frame=single,basicstyle=\small\ttfamily,breaklines=True}

\title{Documentation for laser-propagation}
\author{Jeffrey M. Brown}

\begin{document}
\maketitle
\tableofcontents

\section{Field propagation}
\subsection{Propagation equation}
The propagation of the electric field is represented by the
unidirectional pulse propagation equation written in the spectral
domain as
\begin{equation}
  \label{eq:propagation}
  \frac{\partial \mathcal{E}(z, k_\perp,\omega)}{\partial z} = ik_z(k_\perp,\omega) \mathcal{E}(z, k_\perp,\omega) + \frac{i\omega^2}{2\epsilon_0 c^2 k_{z,n}(k_\perp,\omega)}\left[\mathcal{P}(z, k_\perp,\omega) + i \frac{\mathcal{J}(z, k_\perp,\omega)}{\omega}\right]
\end{equation}
where $k_z(k_\perp,\omega)$ describes how the spectral field
amplitudes propagate linearly in the medium (both diffraction and
linear absorption). The nonlinear properties of the medium are
described by functions representing the nonlinear polarization $P$
and/or nonlinear current $J$. The functional forms of $k_z$, $P$, and
$J$ are determine by the parameter file supplied by user at
runtime. All of the available options are detailed in Section
\ref{sec:parameter-file}.


It's not ideal to solve Equation \eqref{eq:propagation} directly since
it still contains a fast oscillating term resulting from linear
propagation. The is remedied by defining the spectral amplitude
$A(k_\perp,\omega) = E(k_\perp,\omega)\exp[-ik_z z]$, whose evolution is the
govern by the nonlinear sources only
\begin{equation}
  \label{eq:spectral-amplitudes}
  \partial_z \mathcal{A}(z,k_\perp,\omega) = \frac{i\omega^2}{2\epsilon_0 c^2 k_z(k_\perp,\omega)} e^{-ik_z z} \left[\mathcal{P}(z,k_\perp,\omega)  + i \frac{\mathcal{J}(z, k_\perp,\omega)}{\omega}\right]
\end{equation}
The variables $A(k_\perp,\omega)$ are the native variables of the
simulation. Their evolution equation is a large ODE system that is
solved using
\href{https://www.gnu.org/software/gsl/doc/html/ode-initval.html}{GSL's
  ode solver library}. This library provides many methods of solving
ODE's and includes adaptive step control. The ODE solver only requires
the RHS of Eq. \ref{eq:spectral-amplitudes} and some error tolerance
values to control the adaptive stepping algorithm.

\subsection{Algorithm}
At each step in solving Eq. \eqref{eq:spectral-amplitudes}, the solver
invokes a call to the RHS with a proposed step size forward in
$z$. Since nonlinear sources $P$ and $J$ are typically defined in real
space, the RHS function has the job of spectrally transformed
$\mathcal{A}(k_\perp,\omega)$ to $E(r, t)$, calling each nonlinearity,
and transforming back to spectral amplitudes. The RHS function returns
the change in spectral amplitudes
$\partial_z \mathcal{A}(z,k_\perp,\omega)$ to the solver, which
determines whether the proposed step was small enough to evolve the
amplitudes $\mathcal{A}$ within a given error tolerance.

Explicitly, during each step of the ODE solver, the RHS is calculated
the following way:
\begin{enumerate}
\item Linear propagate $\mathcal{A}$ forward by $\Delta z$ ($\omega/v_g$ term
  keeps the pulse centered in the computational box)
  \begin{equation}
    \mathcal{E}(k_\perp,\omega) = A(z, k_\perp,\omega) \exp\left[ i\left(k_z\left(k_\perp,\omega\right) - \omega/v_g\right) \Delta z\right]
  \end{equation}
\item Transform to real space in radial coordinates using the Fourier-Hankel transform
  \begin{equation}
    E(r,t) = \mathrm{Hankel}\{\mathcal{E}(k_\perp)\}\mathrm{FT}\{\mathcal{E}(\omega)\}
  \end{equation}
\item Calculate nonlinearities
  \[P(r,t) = \sum_i P_i[E(r,t)]\]
\item Transform to spectral space
  \[P(k_\perp,\omega) = \mathrm{Hankel}\{P(r)\}\mathrm{FT^{-1}}\{P(t)\}\]
\item Undo the propagation in Step 1
  \begin{equation}
    \mathcal{P}(k_\perp,\omega) = P(k_\perp,\omega) \exp\left[ -i(k_z(k_\perp,\omega) - \omega/v_g) \Delta z\right]
  \end{equation}
\item Finally, multiply the nonlinearity by the coupling factor and
  return to ODE solver
  \begin{equation}
    \partial_z \mathcal{A}(k_\perp,\omega) = \frac{i\omega^2}{2\epsilon_0 c^2 k_z} \mathcal{P}(k_\perp,\omega)
  \end{equation}
\end{enumerate}

\subsection{Computational Grid}
The computational grids in the simulation are three 2D arrays
representing the fields $\mathcal{A}(k_\perp,\omega)$ and $E(r,t)$,
and also the electron density $\rho(r,t)$ in cylindrical coordinates.
There are also some auxillary arrays that are used for computing the
spectral transforms. The spectral transformation of this grid involves a Fourier transform
between time $t$ and frequency $\omega$ and a Hankel transform between
$r$ and $k_\perp$.

To perform the Fourier transform the library
\href{http://fftw.org/}{FFTW} is used. A norm-preserving Hankel
transform was introduced by H. Fisk Johnson in 1986. The radial grid
points are not equally spaced, but are proportional to the zeros of
the Bessel $J_0(\alpha_i) = 0$, and are defined as
$r_i = R_{max} \alpha_i / \alpha_N$ where $N$ is the number of radial
grid points. The values of $k_{\perp,i} = \alpha_i / R_{max}$. With
this coordinate transformation, the Hankel transform is simply a
matrix multiplication with elements
\[H_{mn} = \frac{2J_0(\alpha_n\alpha_m /
    \alpha_N)}{J_1(\alpha_n)J_1(\alpha_m) \alpha_N}\]
%
The Hankel transform is its own inverse so multiplication twice is the
identity: $H H f = f$. This multiplication is speed up by using the
library
\href{http://eigen.tuxfamily.org/index.php?title=Main_Page}{Eigen},
which generates SIMD instructions resulting in a speed up of 30-40\%
in freespace calculations.

The size of the spectral amplitudes $\mathcal{A}$ (number of points in
$k_\perp$ and $\omega$) is always smaller than the size of $E$, since
only positive frequency components of $\mathcal{A}$ are
propagated. This saves quite a bit of memory since the number of
active frequencies $\omega$ are less than half the size of the number
of time points in the grid. Using a capillary waveguide with only a
few number of modes (the number of points in $k_\perp$ greatly reduces
the size of $\mathcal{A}$ as well.  For example, a capillary waveguide
simulation with a real-space grid size of
$Nr\times Nt = 100 \times 16384 = 1.6M$ points is represented by the
spectral amplitude grid-size of
$N\omega\times Nk_\perp = 2199\times 10 = 22k$ points when using 10
capillary modes and only propagating the active frequencies
corresponding to the wavelengths between 250 nm - 3000 nm.

It's good to keep in mind the algorithmic scaling of the two
transformations: FFT and Hankel. The FFT is performed for each radial
grid point and scales as $\mathcal{O}(Nr Nt\log_2 Nt)$. FFT runs most
efficiently with $Nt$ being of a power of two: $2^k$. The Hankel
transform has a different scaling for freespace and capillary
calculations. For freespace calculations a full matrix multiplication
is done for each active frequency $\omega$ and thus scales as
$\mathcal{O}(N\omega Nr^2)$. Due to the reduced $Nk_\perp$ in
capillary simulations this complexity can be reduced to
$\mathcal{O}(N\omega Nk_\perp Nr)$. Since the number of capillary modes
$Nk_\perp$ is usually around 5-10, the Hankel transform in the
capillary simulations scale linearly with the number of radial grid
points.

\section{Parameter file}
\label{sec:parameter-file}
The type of simulation, medium model, choice of nonlinearities, and
output are controlled by a parameter file that is passed to the
executable as an argument on the command line. All values in the
parameter file are assumed to be in SI units. There are some values
that are required in order to run any type of simulation (e.g. setting
up the computational grid, initial conditions, etc), but many values
(e.g. selecting which nonlinearities to include) are optional and only
need to be included in the parameter file if you wish to have them in
the simulation.

\subsection{Syntax}
The syntax of the parameter file is similar to conf or ini styles,
where key-value pairs are grouped into sections. Here's an example:
\begin{lstlisting}
[section1]
key = value 
x = y

# top level comment
[section2]
s = hello
a = 1.1
x = -2e-12   # inline comment
\end{lstlisting}

Whitespace surrounding the keys, values, sections, and $=$ symbol are
ignored. Comments can be placed basically anywhere since all text
after a hash \# symbol is ignored. The keys and values can be read and
runtime to the program by the \texttt{Parameters} class defined in
\texttt{parameters.h} \& \texttt{parameters.cc}.

To read the parameter file \texttt{input}:
\begin{lstlisting}
[time]
N = 1024
time_min = -10e-15
time_max = 50e-15
\end{lstlisting}
instantiate the \texttt{Parameters} class with the filename and
extract the values
\begin{lstlisting}[language=C++]
Parameters::Parameters params("input");
int Nt = params.get<int>("time/N");
double time_min = p.get<double>("time/time_min");
double time_max = p.get<double>("time/time_max");
\end{lstlisting}

Internally, all keys and values are stored as strings and keys have
had their surrounding section prepended to them using a \texttt{/}.
The \texttt{Parameters} class has a template function \texttt{get}
that uses stringstreams to interpret the type of value. This means
that the syntax of the values is limited to readable C++ types:
e.g. \texttt{string}, \texttt{int}, \texttt{double}, etc.


\subsection{Setting up the computational grid}
\subsubsection{[time] (required)}
This section defines the properties of the temporal and spectral
domains of the field. For the temporal domain, the keys to define are
the number of temporal points \texttt{N}, and the minimum
\texttt{time\_min} and maximum \texttt{time\_max} values of the
temporal range.  For efficiency, \texttt{N} should be a power of
2. Additionally, only certain positive frequency components of the
spectral field are propagated. To set the range of active frequencies,
define \texttt{wavelength\_min} and \texttt{wavelength\_max}.

An example of this section is:
\begin{lstlisting}
[time]
N = 2048
time_min = -100e-15
time_max = 200e-15
wavelength_min = 150e-9
wavelength_max = 4e-6
\end{lstlisting}

This defines a temporal box of 2048 points extending from -100fs to
200fs. The active frequencies in the simulation are only the ones that
fall between 150nm and 4 microns.

\subsubsection{[filtering] (optional)}
This section defines temporal and spectral filtering. Filtering is
crude since it amounts to multiplication of the temporal and spectral
fields by $\sin^2$ mask before the forward and inverse Fourier
transformations.  The values defined here along with the ones defined
in sections \texttt{[time]} determine how fast $\sin^2$ ramps go from
0 to 1. Setting a filter min/max value to a value outside the limits
in \texttt{[time]} disables filtering on that particular side.

Reasonable filter values for the time domain in previous section could
be:
\begin{lstlisting}
[filtering]
time_filter_min = -90e-15
time_filter_max = 150e-15
wavelength_filter_min = 250e-9
wavelength_filter_max = 3e-6
\end{lstlisting}

Note: there is no filtering in $r$ or $k_\perp$ which leads to
reflections if the field reaches the radial boundary.

If interested in THz generation, the filter value on the long
wavelength side should be set to a large value:
\begin{lstlisting}
[filtering]
time_filter_min = -90e-15
time_filter_max = 150e-15
wavelength_filter_min = 250e-9
wavelength_filter_max = 1
\end{lstlisting}

\subsubsection{[space] (required)}
This section defines the spatial properties of the radial domain:
\texttt{N} sets the number of radial points, and \texttt{radius\_max}
sets the radial extent of the domain. Since the radial spectral
transform is the Hankel transform, there are no special values for
\texttt{N} that are more efficient, unlike there for the time domain
(due to the FFT).

An example of this section is:
\begin{lstlisting}
[space]
N = 173
radius_max = 10e-3
\end{lstlisting}

This defines a radial domain of 173 points with a maximum extent of 10
mm.

\subsection{Propagtion and ODE step control}
\subsubsection{[propagation] (required)}
This section sets the physical starting and ending distances
(\texttt{starting\_distance} and \texttt{ending\_distance}) of the
simulation, and when data is written to the disk.  The key
\texttt{num\_reports\_cheap} sets the number of cheap diagnostics
(fast to calculate or small in written data size) to perform between
the starting and ending distances, and
\texttt{num\_reports\_expensive} sets the number of expensive
diagnostics. Whether a diagnostic is cheap or expensive is defined by
the how the Observer was added in
\texttt{main.cc:initialize\_observers} via the function
\texttt{main.cc:conditionally\_add}.

An example of this section is:
\begin{lstlisting}
[propagation]
starting_distance = 0
ending_distance = 4
num_reports_cheap = 40
num_reports_expensive = 4
\end{lstlisting}

The laser propagates from 0 to 4 meters and during propagation 40
cheap and 4 expensive diagnostics are performed.

\subsubsection{[ode] (required)}
This section defines the parameters for the ODE solver. Adjustment of
these values has a great impact on the runtime of the simulation,
since they force the solver to take small enough steps to satisfy
these criteria. The values below are passed directly to
\href{https://www.gnu.org/software/gsl/doc/html/ode-initval.html\#c.gsl\_odeiv2\_control\_standard\_new}{GSL's
  ODE step control function}. They form a two-parameter heuristic
value $D_i= \epsilon_{abs} + \epsilon_{rel} |A_i|$ that is compared to
the error calculated by the RK45 method $|Aerr_i|$.

For fast, but perhaps not the most accurate simulations, try:
\begin{lstlisting}
[ode]
absolute_error = 1e10 # or 1e8
relative_error = 1e-4
first_step = 1e-3
\end{lstlisting}
This sets the two error parameters and the first attempted step of the
simulation to be 1 mm.


For production runs where the data should be reasonable converged,
try:
\begin{lstlisting}
[ode]
absolute_error = 1e6
relative_error = 1e-4
first_step = 1e-3
\end{lstlisting}


\subsection{Linear properties}
\subsubsection{[medium] (required)}
\label{sec:medium}
This section defines the linear properties of the medium in the
simulation. The key \texttt{type} controls the functional form of the
linear propagator (basically the formula for $k_z(k_\perp, \omega)$),
and can be set to either \texttt{freespace}, \texttt{capillary}, or
\texttt{diffractionless}.

Selecting \texttt{freespace} sets
\[k_z(k_\perp,\omega) = \sqrt{\left(\frac{n(\omega)\omega}{c}\right)^2 - k_\perp^2}\]

Selecting \texttt{capillary} sets
\[k_z(k_\perp,\omega) = \sqrt{\left(\frac{n(\omega)\omega}{c}\right)^2 - k_\perp^2} + i\alpha(k_\perp)\]
where the absorption $\alpha$ due to the gas cladding interface is
\[\alpha = \frac{1}{2R}\left(\frac{k_\perp c}{\omega}\right)^2 \frac{n_{clad}^2+1}{\sqrt{n_{clad}^2 -1}}\]
where $R$ is the capillary radius and $n_{clad}$ is the index of the
cladding. If \texttt{capillary} is chosen, then the input file must
contain the section \texttt{[capillary]} as described in
\ref{sec:capillary}.

Selecting \texttt{diffractionless} sets
\[k_z(k_\perp,\omega) = \frac{n(\omega)\omega}{c}\] which is useful
for verifying that temporal dispersion is computed properly.

The key \texttt{index} defines the index of refraction function
$n(\omega)$. Currently implemented functions are: \texttt{vacuum},
\texttt{air}, \texttt{argon}, and \texttt{ethanol}. The index can also
be defined using tabulated data from a file that will be linearly
interpolated onto the points $\omega$ of the spectral domain. To do
this set \texttt{index} equal to a filename ending with the
\texttt{.dat} extension and contains three space-separated columns:
$\omega~\Re\{n(\omega)\}~\Im\{n(\omega)\}$.

Lastly, the key \texttt{pressure} is used to increase or decrease the
index of refraction using the Lorentz-Lorenz relation.

An example of freespace propagation in air at a pressure of 1 atm is:
\begin{lstlisting}
[medium]
type = freespace
index = air
pressure = 1
\end{lstlisting}

An example of capillary propagation in argon at 3.3 atm is:
\begin{lstlisting}
[medium]
type = capillary
index = argon
pressure = 3.3
\end{lstlisting}

\subsubsection{[capillary] (optional)}
\label{sec:capillary}
This section defines a capillary waveguide which sets the functional
form of $k_z(k_\perp,\omega)$ (adding absorption from gas-cladding
interface) and also the number of propagating capillary
\texttt{modes}. The index of the cladding is \texttt{cladding}.
\begin{lstlisting}
[capillary]
cladding = 1.45
modes = 5
\end{lstlisting}

\subsection{Initial conditions}
\subsubsection{[laser] (required)}
This section defines the initial laser pulse. The key \texttt{type}
can be \texttt{gaussian}, \texttt{restart}, or \texttt{file}.

An example of a Gaussian beam is:
\begin{lstlisting}
[laser]
type = gaussian
wavelength = 800e-9
length = 20e-15
waist = 4e-3
focus = 2
energy = 1e-3
chirp = 0
phase_deg = 0
delay = 0
\end{lstlisting}
Here the laser has a envelope shape of Gaussian in space and
time. It's central wavelength is 800 nm, FWHM temporal length is 20
fs, 1/e$^2$ radius is 4 mm, is focused at 2 m, has energy of 1 mJ, and
has zero chirp, phase and temporal delay. Note: the
\texttt{starting\_distance} in \texttt{[propagation]} can be used to
linear propagate forward in $z$ assuming Gaussian beam propagation in
space and a Gaussian-like propagation in time through a dispersive
media (gvd and chirp are taken into account). This is useful for only
performing a nonlinear simulation near the focus, while propagating
linearly through a medium beforehand.

An example of restarting from an old simulation:
\begin{lstlisting}
[laser]
type = restart
filename = A004.dat
wavelength = 800e-9
\end{lstlisting}
The \texttt{filename} must point to a spectral field file that was
generated from a previous simulation, and you must specify the central
wavelength (used for calculating values, such as $n0$, $gvd$,
$P_{cr}$). Note: you might want to set the \texttt{starting\_distance}
in \texttt{[propagation]} to match the physical distance of the
spectral field file.

An example of starting from a binary space-time field file:
\begin{lstlisting}
[laser]
type = file
filename = init_field.dat
wavelength = 800e-9
\end{lstlisting}
Note: the field $E(r,t)$ in \texttt{init\_field.dat} must be stored in
a binary file of doubles (8 bytes) where the real and imaginary values
are stored as [real0 imag0 real1 imag1 ...]. The time axis $t$ is the
fast axis (convention of C, opposite of Fortran). The sampling of $r$
and $t$ must match the values are set in the sections \texttt{[time]}
and \texttt{[space]}.


\subsection{Output data}
\subsubsection{[results] (optional)}
This section defines which data is written to which files during
propagation.  All of the entries in this section are
optional. Commenting or removing any line below results in no data
being written for that particular item.
\begin{lstlisting}
[results]
# field and density filename patterns and their associated distances
temporal_field = E
spectral_field = A
electron_density = Rho
distance = distance.dat

# coordinates
time = time.dat
radius = radius.dat
omega = omega.dat
kperp = kperp.dat
wavelength = wavelength.dat

# values along propagation
energy = energy.dat
max_intensity = max_intensity.dat
max_density = max_density.dat
\end{lstlisting}

The keys are fixed by the program, but the values can be changed by
the user.  The only special values are the ones for the field and
density files (\texttt{temporal\_field = E}, \texttt{spectral\_field =
  A}, and \texttt{electron\_density = Rho}) where their filename values
contain no extension (e.g. dat or txt). This is because their
filenames will be enumerated (e.g. E000.dat, E001.dat, E002.dat, etc.)
and the distances at which they are written out is contained in
\texttt{distance}.

\subsection{Nonlinear properties}
\subsubsection{[kerr] (optional)}
Defining this section adds Kerr nonlinearity to the simulation using
\[P_{\mathrm{NL}}(r, t) = \epsilon_0 \chi^{(3)} E^3(r,t)\] where
$\chi^{(3)} = \frac43 \epsilon_0 c n_2 n_0^2$. The only value to set is the
nonlinear coefficient $n_2$.
\begin{lstlisting}
[kerr]
n2 = 8e-24
\end{lstlisting}
Note: The value of $\chi^{(3)}$ is multiplied by the value of
\texttt{pressure} that is set in section \texttt{[medium]}.

\subsubsection{[ramankerr] (optional)}
Defining this section add the Raman-Kerr nonlinearity to the
simulation as:
\[P_{\mathrm{NL}}(r,t) = \epsilon_0 \chi^{(3)} \left[(1-f_R) E^2(r,t)
    + f_R\int_{-\infty}^t R(t-t') E^2(t')\right] E(r,t)\] where $f_R$
is the Raman fraction, $R(t) = R_0 \exp(-\Gamma t/r) \sin(\Lambda t)$,
$R_0 = (\Gamma^2/4 + \Lambda^2) / \Lambda$,
$\chi^{(3)} = \frac43 \epsilon_0 c n_2 n_0^2$.  The required keys are
\texttt{n2} for the nonlinear coefficient, \texttt{fraction} for the
proportion of $n2$ associated with the delayed Raman response, and
\texttt{gamma} \& \texttt{lambda} for the frequency parameters of
$R(t)$.

This equation is solved using the exponential time differencing
method. The discrete formula for the time integration for array index
$j$ from 0 to $Nt$ is
\[P_{\mathrm{NL}}[j] = \epsilon_0 \chi^{(3)} \left[(1-f_R) E[j]^2 + f_R Q[j]\right] E[j] \]
with an initial condition for $j = 0$
\[Q[0] = R_0 \frac{\Delta t}{2} E[0]^2\]
and for $j \ge 1$
\[Q[j] = \Im\bigg\{e^{(-\Gamma/2 + i\Lambda)\Delta t} Q[j-1] + R_0\frac{\Delta t}{2} \left[E[j]^2 + e^{(-\Gamma/2 + i\Lambda)\Delta t} E[j-1]^2\right]\bigg\}\]

An example of this section is:
\begin{lstlisting}
[ramankerr]
n2 = 22.3e-24
fraction = 0.65
gamma = 10e12
lambda 16e12
\end{lstlisting}


Note: the value of
$\chi^{(3)}$ is multiplied by the value of \texttt{pressure} that is
set in section \texttt{[medium]}.

\subsubsection{[ionization] (optional)}
Defining this section adds a rate of ionization which populates the
electron density $\rho(r,t)$ at each step of the simulation according to
\[\frac{\partial \rho}{\partial t} = W(E(t)) (f\rho_{nt}-\rho)\]
where $W$ is the rate of ionization and $\rho_{nt}$ is the
\texttt{density\_of\_neutrals}.  This equation is solved using the
exponential time differencing method
\[\rho(t+\Delta t) = \exp\left[-\int_t^{t+\Delta t} W(E(t')) dt'\right]\left\{\rho(t) + \frac{\Delta t}{2} \rho_{nt}W(E(t))\right\} + \frac{\Delta t}{2} \rho_{nt} W(E(t+\Delta t))\]

If the keys \texttt{generate} and \texttt{formula} are present the
rate of ionization is calculated for the given central laser
wavelength (defined in \texttt{[medium]}) and
\texttt{ionization\_potential}. The value of \texttt{generate} is the
filename to which the rates are written (two-column data corresponding
to the intensity [W/m$^2$] and rate [1/s]). The options for
\texttt{formula} are \texttt{adk}, \texttt{mpi}, \texttt{tunnel},
\texttt{yudin}, and \texttt{ilkov}.

The rates can also be read from a file by defining the key
\texttt{filename}. This file should contain two-column data
corresponding to the intensity [W/m$^2$] and rate [1/s].

Defining \texttt{[ionization]} also adds nonlinear absorption
\[J_{\mathrm{NA}}(t) = \frac{W(E(t))}{I(t)} U_i \left(f\rho_{nt} -
    \rho(t)\right) \epsilon_0 c E(t)\] where $U_i$ is the
\texttt{ionization\_potential} and $f$ is the fraction of neutrals
that can be ionized (\texttt{ionizing\_fraction}).

Defining \texttt{[ionization]} also adds effects due to plasma according to
\[\frac{\partial J_{\mathrm{PL}}}{\partial t} + \frac{J_{\mathrm{PL}}(t)}{\tau_c} = \frac{e^2}{2 m_e} \rho(t) E(t)\]
where $\tau_c$ is the \texttt{collision\_time}. This equation is
solved using the exponential time differencing method:
\[J_{\mathrm{PL}}(t+\Delta t) = e^{-\Delta t/\tau_c}\left\{J_{\mathrm{PL}}(t) + \frac{\Delta t e^2}{2 m_e}\rho(t)E(t)\right\} + \frac{\Delta t e^2}{2 m_e} \rho(t+\Delta t) E(t+\Delta t)\]

\begin{lstlisting}
[ionization]
filename = ../../data/ionization-argon.txt
ionizing_fraction = 1
density_of_neutrals = 2.5e25
collision_time = 190e-15
ionization_potential = 2.5250303e-18
\end{lstlisting}

\subsubsection{[argon] (optional)}
\textbf{This is an experimental feature!} Adding the section
\texttt{[argon]} to the parameter file initializes a realistic SAE
model for argon. This model is based on the work from ``Calculations
of strong field multiphoton processes\ldots'', Schafer et al., AIP
Conf. Proc. \textbf{525}, 45 (2000).

There is an atom placed at each radial point, so the number of quantum
systems in the simulation is equal to the value of \texttt{N} in
\texttt{[space]}. The time-dependent nonlinear polarization (from the
dipole moment) and ionization probability are computed for each atom
as it interacts with the laser field.

Since the time to compute the response of a single atom is anywhere
from 20-120 s (depending on the values that are chosen for the quantum
system), it's highly recommended that the parallel MPI version of the
executable \texttt{main-mpi.out} is used. Since there is no
interaction between the individual argon atoms, parallelization is
straightforward. The laser simulation run on processor 0, as well as a
single argon atom for radial grid point 0. Every other processor holds
a single argon atom for the radial grid points 1 to N.  When the
electron density or nonlinear polarization is calculated, the laser
field for each radial point is distributed to every processor using
\texttt{MPI\_Scatter}. Everyone calculates the electron density and
nonlinear dipole moment, then sends their results back to processor 0
using \texttt{MPI\_Gather}. Since there is currently the limitation of
one quantum system per radial grid point, it is necessary to launch
the executable with the same number of threads as radial grid points
defined in \texttt{[space]}. If this is not the case, the program
signals an error.

A good set of parameters to start with for Argon gas at 15 C, 1 atm
are the following:
\begin{lstlisting}
[argon]
Nr = 300
Nl = 21
Nmask = 100
step_size = 1
potential = short-range-potential.dat
ionization_box_size = 40
density_of_neutrals = 2.55e25
collision_time = 190e-15
ionization_potential = 2.5250303e-18
\end{lstlisting}
where \texttt{Nr} is the number of grid points in radius for the
wavefunction (the spacing in $r$, $\Delta r$ is fixed at 0.25 in
atomic units), \texttt{Nl} is the number of angular momentum states,
\texttt{Nmask} is the number of radial grid points for the absorber,
\texttt{step\_size} is the time scale (in atomic units) on which the
wave function evolves, \texttt{potential} is the filename that
contains the short range potentials of argon for $l=0,1,2$ states. The
parameter \texttt{ionization\_box\_size} specifies the number of
radial grid points above which the wavefunction is considered ionized
(i.e. the part of the wavefunction that is outside of this sphere is
considered part of the free electron probability). Finally, for
scaling the single atom response up to a macroscopic response, it's
necessary to define the density of neutrals.

Nonlinear effects from plasma are also added and contribute a current:
\[\frac{\partial J_{\mathrm{PL}}}{\partial t} + \frac{J_{\mathrm{PL}}(t)}{\tau_c} = \frac{e^2}{2 m_e} \rho(t) E(t)\]
where $\tau_c$ is the \texttt{collision\_time}. Nonlinear absorption
due to ionization is also added using:

Nonlinear effects from plasma are also added:
\[J_{\mathrm{NA}}(t) = \frac{W(E(t))}{I(t)} U_i \left(f\rho_{nt} -
    \rho(t)\right) \epsilon_0 c E(t)\]
where $U_i$ is the \texttt{ionization\_potential}.



\end{document}
